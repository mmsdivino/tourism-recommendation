O avanço da tecnologia e o aumento do uso da Internet gera um crescimento da quantidade de informações na Web. A expansão de usuários de \textit{smartphones} e em redes sociais proporciona uma grande produção de dados. Na presença dessas informações, nem sempre é possível diferenciar o que é útil, existindo a necessidade de filtrar para encontrar dados relevantes. Sistemas de Recomendação vem como uma das resoluções para esse problema. Através de dados de usuários e itens, o sistema tem a capacidade de recomendar itens de acordo com as preferências de um determinado usuário.

Este quadro também acontece em áreas do ramo turístico. Normalmente o turista pode ter dificuldade em construir um itinerário de modo que possa visitar pontos turísticos que estejam dentro das suas preferências. Diante disso, este trabalho visa otimizar o uso de localização geográfica para recomendação de pontos turísticos em aplicações de apoio ao turismo. Este projeto apresenta um sistema de recomendação que sugere uma rota com os pontos turísticos mais relevantes ao usuário, de acordo com os seus interesses. As recomendações são baseadas em avaliações de outros usuários para os pontos turísticos da cidade de Salvador. Tecnicamente, os modelos de recomendação utilizam métodos baseados na similaridade entre os item e baseado no conteúdo dos pontos de interesses.

Os resultados mostram que é possível realizar recomendações de pontos turísticos, considerando as opções do usuário. As métricas de \textit{Precision} e \textit{Recall} tiveram uma melhora de aproximadamente 75\% e 90\% respectivamente entre os dois modelos de recomendação utilizados. Este trabalho pode ser considerado o início da resolução de um problema que é frequente a diversos turistas diante de uma grande quantidade de opções de pontos de interesse.

\begin{keywords}
Sistemas de Recomendação, Turismo, Rotas, Localização
\end{keywords}